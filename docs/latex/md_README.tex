In this second assignment, we were asked to develop a software architecture for controlling a robot within a specific environment. More in details, our software will rely on two different packages\+:


\begin{DoxyEnumerate}
\item the move\+\_\+base pakage
\item the gmapping package
\end{DoxyEnumerate}

The first one, will take care of the localization of our robot within the enviroment, whereas the second one will plan the motion.

\subsection*{User request}

The architecture will provide a way to get the {\bfseries user request}, and will make the robot capable of executing one of the following behaviors (depending on the user’s input)\+:


\begin{DoxyEnumerate}
\item Move randomly in the environment, by choosing 1 out of 6 possible target position \tabulinesep=1mm
\begin{longtabu} spread 0pt [c]{*2{|X[-1]}|}
\hline
\rowcolor{\tableheadbgcolor}{\bf Position }&{\bf Coordinates  }\\\cline{1-2}
\endfirsthead
\hline
\endfoot
\hline
\rowcolor{\tableheadbgcolor}{\bf Position }&{\bf Coordinates  }\\\cline{1-2}
\endhead
Position 1 &(-\/4,-\/3) \\\cline{1-2}
Position 2 &(-\/4,2) \\\cline{1-2}
Position 3 &(-\/4,7) \\\cline{1-2}
Position 4 &(5,-\/7) \\\cline{1-2}
Position 5 &(5,-\/3) \\\cline{1-2}
Position 6 &(5,1) \\\cline{1-2}
\end{longtabu}

\item Directly ask the user for the {\bfseries next} target position (checking that the position is one of the possible six) and reach it)
\item Start following the external walls
\item Stop in the last position
\item (optional) change the planning algorithm to {\itshape dijkstra} (move\+\_\+base) to the {\ttfamily bug0}
\end{DoxyEnumerate}

\section*{How the architecture is structured}

Please, find in the following sections a brief summary of the various R\+OS messages, services, parameters, and other sub-\/structures used to realise this project

\subsection*{Messages}

\subsubsection*{Published messages}

In the list below, the {\bfseries published messages} of the whole architecture are reported\+:


\begin{DoxyItemize}
\item {\bfseries actionlib\+\_\+msgs/\+Goal\+ID}\+: It publishes the messages to the $\ast$$\ast$/move\+\_\+base/cancel$\ast$$\ast$ topic in order to remove a target as it is considered reached. Indeed, it allows to easily avoid the inadvertent overlying of robot {\itshape distinct} behaviours
\item {\bfseries geometry\+\_\+msgs/\+Twist}\+: It publishes the messages to the $\ast$$\ast$/cmd\+\_\+vel$\ast$$\ast$ topic. Moreover, it is used\+:
\begin{DoxyEnumerate}
\item for setting both the robot linear and and angular velocity
\item for halting the robot.
\end{DoxyEnumerate}
\item {\bfseries move\+\_\+base\+\_\+msgs/\+Move\+Base\+Action\+Goal}\+: It publishes the messages to the $\ast$$\ast$/move\+\_\+base/goal$\ast$$\ast$ topic. Moreover, it is used for setting the move\+\_\+base goal that the robot has to achieve
\end{DoxyItemize}

\subsubsection*{Subscribed messages}

In the list below, the {\bfseries subscribed messages} of the whole architecture are reported\+:


\begin{DoxyItemize}
\item {\bfseries geometry\+\_\+msgs/\+Point}\+: It defines the robot position, expressed as a {\bfseries Point}
\item {\bfseries nav\+\_\+msgs/\+Odometry}\+: It provides the current robot position by means of the $\ast$$\ast$/\+Odom$\ast$$\ast$ topic
\item {\bfseries sensor\+\_\+msgs/\+Laser\+Scanl} It provides the real-\/time laser output through the $\ast$$\ast$/scan$\ast$$\ast$ topic
\end{DoxyItemize}

\subsection*{Services}

As far as services are concerned, I have here-\/reported a quick explanation of their purposes\+:


\begin{DoxyEnumerate}
\item Within the \href{https://github.com/fedehub/final_assignment/tree/main/scripts/random_srv.py}{\tt random\+\_\+srv.\+py} script, there is the {\ttfamily /random\+\_\+srv} service. Its purpose is to choose between six different coordinate positions
\item Within the \href{https://github.com/fedehub/final_assignment/tree/main/scripts/go_to_point_service_m.py}{\tt go\+\_\+to\+\_\+point\+\_\+service\+\_\+m.\+py} script, there is the {\ttfamily /go\+\_\+to\+\_\+point\+\_\+switch} service. Its purpose is to {\itshape activate/deactivate} the {\bfseries go-\/to-\/point} behaviour
\item Within the \href{https://github.com/fedehub/final_assignment/tree/main/scripts/ui.py}{\tt ui.\+py} script, we find the {\ttfamily /ui} service. its purpose consists in providing a tool for selecting one out of the {\bfseries five} possible robot states. In the table below, it is possible to spot what, the different states implement\+:
\end{DoxyEnumerate}

\tabulinesep=1mm
\begin{longtabu} spread 0pt [c]{*2{|X[-1]}|}
\hline
\rowcolor{\tableheadbgcolor}{\bf \+\_\+state\+\_\+ }&{\bf {\itshape Coordinates}  }\\\cline{1-2}
\endfirsthead
\hline
\endfoot
\hline
\rowcolor{\tableheadbgcolor}{\bf \+\_\+state\+\_\+ }&{\bf {\itshape Coordinates}  }\\\cline{1-2}
\endhead
$\ast$$\ast$state 1$\ast$$\ast$ &move randomly \\\cline{1-2}
$\ast$$\ast$state 2$\ast$$\ast$ &target position \\\cline{1-2}
$\ast$$\ast$state 3$\ast$$\ast$ &walls following \\\cline{1-2}
$\ast$$\ast$state 4$\ast$$\ast$ &stop \\\cline{1-2}
$\ast$$\ast$state 5$\ast$$\ast$ &bug algorithm \\\cline{1-2}
\end{longtabu}
\begin{quote}
{\itshape R\+E\+M\+A\+RK\+:} If {\ttfamily state 5} is chosen, I have implemented a sort of {\itshape timer} which allows the user to pick up another new target by means of a user-\/interface if an inserted target is not reachable. Once the target position has been reached or once the settled timer\textquotesingle{}s time (20 secs, in my implementation) has expired, the program exits the bug state \end{quote}



\begin{DoxyEnumerate}
\item Within the \href{https://github.com/fedehub/final_assignment/tree/main/scripts/user_interface.py}{\tt user\+\_\+interface.\+py} script, we find the {\ttfamily /user\+\_\+interface service}. Its purpose basically consists in providing a tool for selecting the {\itshape x,y coordinates} of the next desired robot target position, once the previous target has been reached. Moreover, this is the {\itshape user-\/interface} exploited by the {\bfseries bug algorithm}
\item Within the \href{https://github.com/fedehub/final_assignment/tree/main/scripts/wall_follower_service_m.py}{\tt wall\+\_\+follower\+\_\+service\+\_\+m.\+py} script, we find the {\ttfamily /wall\+\_\+follower\+\_\+switch} service. Its purpose is to {\itshape activate/deactivate} the {\bfseries wall-\/follower} behaviour
\item Within the \href{https://github.com/fedehub/final_assignment/tree/main/scripts/wall_follower_bug.py}{\tt wall\+\_\+follower\+\_\+bug.\+py} script, we find the {\ttfamily /wall\+\_\+follower\+\_\+bug} service. Its purpose is to {\itshape activate/deactivate} the {\bfseries wall-\/follower} behaviour. \begin{quote}
{\itshape R\+E\+M\+A\+RK\+:} This is the service exploited by the {\ttfamily bug algorithm}! \end{quote}

\item Within the \href{https://github.com/fedehub/final_assignment/tree/main/scripts/bug_m.py}{\tt bug\+\_\+m.\+py} script, we find the {\ttfamily /bug\+\_\+alg} service. Its purpose is to {\itshape activate/deactivate} the {\bfseries bug algorithm} behaviour (through a boolean value)
\end{DoxyEnumerate}

\subsection*{Nodes}

Within the \href{https://github.com/fedehub/final_assignment/tree/main/scripts}{\tt scripts} folder, the following nodes are available\+:


\begin{DoxyEnumerate}
\item \href{https://github.com/fedehub/final_assignment/tree/main/scripts/master_node.py}{\tt master\+\_\+node.\+py} It implements the structure of the entire architecture. It handles the simulation structure and it provides a way for checking the robot state. Then, it triggers the required service.
\item \href{https://github.com/fedehub/final_assignment/tree/main/scripts/user_interface.py}{\tt user\+\_\+interface.\+py} As its name suggests, It provides a service, exploited by the {\bfseries bug algorithm},for letting the user choose which {\bfseries target position} the robot should achieve.
\item \href{https://github.com/fedehub/final_assignment/tree/main/scripts/ui.py}{\tt ui.\+py} It provides a user-\/interface in which the user has to select 1 out of 5 available states. Then, once inserted the chosen state, a \char`\"{}valid state\char`\"{} {\bfseries message} appears, so that the user can save his/her choice by typing the string \char`\"{}done\char`\"{} \begin{quote}
{\itshape R\+E\+M\+A\+RK\+:} When entering the \char`\"{}done\char`\"{} string or other strings, remember to add the quote marks \char`\"{}\char`\"{} \end{quote}

\item \href{https://github.com/fedehub/final_assignment/tree/main/scripts/wall_follower_bug.py}{\tt wall\+\_\+follower\+\_\+bug.\+py} It provides a service for avoiding the collision between our robot and the neighboring walls while performing the target achievement \begin{quote}
{\itshape R\+E\+M\+A\+RK\+:} We need to distinguish between {\itshape \hyperlink{wall__follower__bug_8py}{wall\+\_\+follower\+\_\+bug.\+py}} and {\itshape \hyperlink{wall__follower__service__m_8py}{wall\+\_\+follower\+\_\+service\+\_\+m.\+py}} because even if they share the same code structure, when calling the latter service, both \textquotesingle{}bug algorithm\textquotesingle{} and its {\itshape \hyperlink{namespacewall__follower__service__m}{wall\+\_\+follower\+\_\+service\+\_\+m}} are disabled! \end{quote}

\item \href{https://github.com/fedehub/final_assignment/tree/main/scripts/wall_follower_service_m.py}{\tt wall\+\_\+follower\+\_\+service\+\_\+m.\+py} As its name suggests, it provides a service for simulating the wall following behaviour
\item \href{https://github.com/fedehub/final_assignment/tree/main/scripts/go_to_point_service_m.py}{\tt go\+\_\+to\+\_\+point\+\_\+service\+\_\+m.\+py} It provides a service for making the {\bfseries bug algorithm$\ast$$\ast$$\ast$ capable of implementing the $\ast$$\ast$go-\/to-\/point} behaviour
\item \href{https://github.com/fedehub/final_assignment/tree/main/scripts/random_srv.py}{\tt random\+\_\+srv.\+py} It provides a straightforward service, for setting a {\bfseries random position} which the robot will achieve
\item \href{https://github.com/fedehub/final_assignment/tree/main/scripts/bug_m.py}{\tt bug\+\_\+m.\+py} It provides a service for carrying out the {\bfseries bug algorithm} behaviour
\end{DoxyEnumerate}

\subsection*{Parameters}

They are contained in the \href{https://github.com/fedehub/final_assignment/tree/main/launch}{\tt launch} folder. More specifically\+:


\begin{DoxyItemize}
\item Within the \href{https://github.com/fedehub/final_assignment/blob/main/launch/node_master.launch}{\tt node\+\_\+master.\+launch} file, I have initialised\+:
\begin{DoxyItemize}
\item {\ttfamily des\+\_\+pos\+\_\+x} and {\ttfamily des\+\_\+pos\+\_\+y} for allocating the {\itshape x,y coordinates} of the target the user has chosen (so that the robot can reach them)
\item {\ttfamily state} for individuating the current state (between {\itshape 1 and 5}) of the robot
\item {\ttfamily change\+\_\+state} for initialising at {\itshape zero} the current state of our robot. It has been proven very useful during the implementation of the {\bfseries bug algorithm}
\end{DoxyItemize}
\end{DoxyItemize}

\section*{How to launch}


\begin{DoxyEnumerate}
\item Firstly, create a folder named \+\_\+\char`\"{}final\+\_\+assignment\char`\"{}\+\_\+
\item Within the aforementioned folder, open the terminal and run\+:
\end{DoxyEnumerate}


\begin{DoxyCode}
1 git clone https://github.com/fedehub/final\_assignment/
\end{DoxyCode}

\begin{DoxyEnumerate}
\item Then, to launch the {\itshape 3D visualizer Rviz} and the {\itshape 3D simulator Gazebo} please run the command 
\begin{DoxyCode}
1 roslaunch final\_assignment simulation\_gmapping.launch
\end{DoxyCode}

\item To launch the {\itshape move\+\_\+base.\+launch}, digit\+:
\end{DoxyEnumerate}


\begin{DoxyCode}
1 roslaunch final\_assignment move\_base.launch
\end{DoxyCode}

\begin{DoxyEnumerate}
\item To conclude with, digit\+:
\end{DoxyEnumerate}


\begin{DoxyCode}
1 roslaunch final\_assignment.launch
\end{DoxyCode}


\subsection*{Documentation}

The documentation of this project, obtained by means of {\bfseries Doxy\+Gen} is visible, within the \href{https://github.com/fedehub/final_assignment/tree/main/docs}{\tt docs} folder

\subsection*{System limitation\textquotesingle{}s and possible Improvements}

\subsection*{Release History}


\begin{DoxyItemize}
\item {\ttfamily 0.\+1.\+0}
\begin{DoxyItemize}
\item The first proper release
\end{DoxyItemize}
\item {\ttfamily 0.\+0.\+1}
\begin{DoxyItemize}
\item Work in progress
\end{DoxyItemize}
\end{DoxyItemize}

\subsection*{Meta}

Federico Civetta– s4194543 – \href{mailto:fedeunivers@gmail.com}{\tt fedeunivers@gmail.\+com}

Distributed under none licence

\href{https://github.com/fedehub}{\tt https\+://github.\+com/fedehub}

\subsection*{Contributing}


\begin{DoxyEnumerate}
\item Fork it (\href{https://github.com/fedehub/final_assignment/fork}{\tt https\+://github.\+com/fedehub/final\+\_\+assignment/fork})
\item Create your feature branch ({\ttfamily git checkout -\/b feature/foo\+Bar})
\item Commit your changes ({\ttfamily git commit -\/am \textquotesingle{}Add some foo\+Bar\textquotesingle{}})
\item Push to the branch ({\ttfamily git push origin feature/foo\+Bar})
\item Create a new Pull Request 
\end{DoxyEnumerate}